\usepackage[utf8]{inputenc}
\usepackage{biblatex}
\usepackage{hyperref}
\usepackage{color}
\usepackage{amsmath, amsthm, amsfonts, amssymb, mathrsfs}
\usepackage{tikz, tikz-cd}

\newcommand{\Ob}{\mathrm{Ob}}
\newcommand{\Fl}{\mathrm{Fl}}
\newcommand{\Hom}{\mathrm{Hom}}

\newcommand{\namedcat}[1]{\mathsf{#1}}
\newcommand{\Cat}{\namedcat{Cat}}

\newcommand{\category}[1]{\mathscr{#1}}
\newcommand{\E}{\category E}
\newcommand{\F}{\category F}
\newcommand{\oxi}{\overline \xi}
\newcommand{\ozeta}{\overline \zeta}
\newcommand{\oeta}{\overline \eta}


% theorem environments
\theoremstyle{plain}
\newtheorem{thm}{Theorem}
\newtheorem{lem}[thm]{Lemma}
\newtheorem{prop}[thm]{Proposition}
\newtheorem{cor}[thm]{Corollary}
\newtheorem{conj}[thm]{Conjecture}
\newtheorem{defn}[thm]{Definition}
\newtheorem{expl}[thm]{Example}
%
\newtheorem*{thm*}{Theorem}
\newtheorem*{lem*}{Lemma}
\newtheorem*{prop*}{Proposition}
\newtheorem*{cor*}{Corollary}
\newtheorem*{defn*}{Definition}
\newtheorem{expl*}[thm]{Example}
%
\theoremstyle{remark}
\newtheorem{rem}[thm]{Remark}
\newtheorem{note}[thm]{Note}
